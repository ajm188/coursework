\documentclass[12pt]{article}

\usepackage{amsmath}
\usepackage{amsthm}

\title{EECS 440 HW5}
\author{Andrew Mason}

\begin{document}
\maketitle

\begin{enumerate}
  \item
    % 1
    It may be beneficial to overfit if you know that real data will have about
    the same distribution as the training data. That is, the closer the
    training data is to ``real world'' data, the more beneficial overfitting
    is. The reason to avoid overfitting is to prevent the model from performing
    poorly later due to high variance between the training data and real
    samples. However, if there is no (or little) variance, then an overfit
    model will perform extremely well both in evaluation and on real samples.\\
  \item
    % 2
    Sure; this approach is strikingly similar to $n$-fold cross validation. The
    test sets are drawn independently of each other, which is the most important
    aspect of cross validation. The only issue that this approach might cause is
    that the train and test sets are the same size, where you would probably get
    better results if you trained on larger sets.\\
  \item
    % 3
  \item
    % 4
  \item
    % 5
\end{enumerate}
\end{document}
