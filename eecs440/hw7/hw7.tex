\documentclass[12pt]{article}

\usepackage{mathtools}

\title{EECS 440 HW7}
\author{Andrew Mason}

\begin{document}
\maketitle

\begin{enumerate}
  % 1
  \item
    We can view this as if we had done a single test with all $N$ sets of
    examples. Then, the classifier has made $\sum_{i=0}^{N}r_i$ errors out of
    $\sum_{i=0}^Nn_i$ total examples, so the MLE estimate of the true error
    rate is
    \begin{equation}
      \frac{\sum_{i=0}^Nr_i}{\sum_{i=0}^Nn_i}
    \end{equation}
  % 2
  \item
    To establish the difference between A and B at 95\% confidence, we need
    that $0\notin\left(0.1\pm1.96\sigma\right)$.

    In other words,
    \begin{equation}
      \begin{split}
        0&<0.1-1.96\sigma\\
        \sigma&<0.05\\
      \end{split}
    \end{equation}

    So, looking at the variance, we need:
    \begin{equation}
      \begin{split}
        0.05&>\left[\frac{e_A\left(1-e_A\right)}{n}+\frac{e_B\left(1-e_B\right)}{n}\right]^2\\
        0.05n^2&>e_A\left(1-e_A\right)+e_B\left(1-e_B\right)\\
      \end{split}
    \end{equation}

    Also, since $e_A=e_B+0.1$, we have $e_A^2=e_B^2+0.2e_B+0.01$. Now:
    \begin{equation}
      \begin{split}
        0.05n^2&>e_A\left(1-e_A\right)+e_B\left(1-e_B\right)\\
        0.05n^2&>e_B+0.1-e_B^2-0.2e_B-0.01+e_B-e_B^2\\
        0.05n^2&>-2e_B^2+1.98e_B+0.09\\
        n^2&>\frac{-2e_B^2+1.98e_B+0.09}{0.05}\\
        n&>\sqrt{\frac{-2e_B^2+1.98e_B+0.09}{0.05}}\\
      \end{split}
    \end{equation}
  % 3
  \item
    Let $r$ be the number of errors made by Professor X's classifier
    ($\mu_X, \sigma_X$ are the mean and standard deviation, respectively). Then
    the C\% confidence interval of Professor X's classifier is:
    \begin{equation}
      \begin{split}
        CI&=\left(\mu_X\pm z_C\sigma_X\right)\\
        &=\left(\frac{r}{n}\pm z_C\sqrt{\frac{r\left(1-r\right)}{n}}\right)\\
        &=\left(\frac{r\pm z_C\sqrt{n\left(1-r\right)}}{n}\right)\\
      \end{split}
    \end{equation}

    Then, with $E_Y$ as the error rate:
    \begin{equation}
      \begin{split}
        Pr\left(E_Y\in CI\right)&
          =Pr\left(E_Y<\frac{r+z_C\sqrt{n\left(1-r\right)}}{n}\right)-
          Pr\left(E_Y<\frac{r-z_C\sqrt{n\left(1-r\right)}}{n}\right)\\
        &=\frac{r+z_C\sqrt{n\left(1-r\right)}}{n}-\frac{r-z_C\sqrt{n\left(1-r\right)}}{n}\\
        &=\frac{r+z_C\sqrt{n\left(1-r\right)}-r+z_C\sqrt{n\left(1-r\right)}}{n}\\
        &=\frac{2z_C\sqrt{n\left(1-r\right)}}{n}\\
      \end{split}
    \end{equation}
  % 4
  \item
    First, we will assume that Band Wagon over hears the percentages for the
    confidence interval and the lower and upper bounds of the confidence
    interval. For example, Bob says ``My C\% CI was (l, u)''.

    Now, define $C_B,l_B,u_B$ to be the confidence percentage, lower and upper
    bounds for Bob, and $C_N,l_N,u_N$ defined similarly for Nan. Then, let us
    consider how to derive $\mu_B,\sigma_B,\mu_N,\sigma_N$, which will be the
    mean and standard deviation of Bob's and Nan's tests, respectively.

    \begin{equation}
      \begin{split}
        (l_B,u_B)&=(\mu_B\pm C_B\sigma_B)\\
        u_B-l_B&=2C_B\sigma_B\\
        \sigma_B&=\frac{u_B-l_B}{2C_B}\\
      \end{split}
    \end{equation}

    Then,
    \begin{equation}
      \begin{split}
        \mu_B&=l_B+C_B\sigma_B\\
        \mu_B&=l_B+C_B\frac{u_B-l_B}{2C_B}\\
        \mu_B&=l_B+\frac{u_B-l_B}{C_B}\\
        \mu_B&=\frac{C_Bl_B+u_B-l_B}{C_B}\\
        \mu_B&=\frac{\left(C_B-1\right)l_B+u_B}{C_B}\\
      \end{split}
    \end{equation}

    So $\mu_N=\frac{\left(C_N-1\right)l_N+u_N}{C_N}$ and
    $\sigma_N=\frac{u_N-l_N}{2C_N}$ similarly.

    Now, Band Wagon can compute his $\mu_W,\sigma_W$ as follows:
    \begin{equation}
      \begin{split}
        \mu_W&=\mu_B-\mu_N
      \end{split}
    \end{equation}
    \begin{equation}
      \begin{split}
        \sigma_W^2&=\text{Var}\left(W\right)\\
        &=\frac{\mu_B\left(1-\mu_B\right)}{n}+
          \frac{\mu_N\left(1-\mu_N\right)}{n}\\
        &=\frac{\mu_B\left(1-\mu_B\right)+\mu_N\left(1-\mu_N\right)}{n}\\
      \end{split}
    \end{equation}
    and he can report a confidence interval of
    $\left(\mu_W\pm z_W\sigma_W\right)$, where $z_W$ is the value
    corresponding to the confidence percentage of his choosing.\\
  % 5
  \item
\end{enumerate}
\end{document}
