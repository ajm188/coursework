\documentclass[12pt]{article}

\usepackage{amsmath}
\usepackage{amsthm}
\usepackage[margin=1in]{geometry}
\usepackage{mathtools}

\title{EECS 477 HW2}
\author{Andrew Mason}

\begin{document}
\maketitle

\begin{enumerate}
  % #1
  \item
    As in HW1, begin by putting the problem in standard form:\\

    \begin{equation}
      \begin{split}
        \text{min }& -3x_1 + 5x_2 \\
        \text{s.t. }& 4x_1 + 5x_2 - s = 3 \\
        & 6x_1 - 6x_2 = 7 \\
        & x_1 + 8x_2 + t = 20 \\
        & x_1, x_2, s, t\geq 0 \\
      \end{split}
    \end{equation}

    Again, from HW1, we have that
    $x^*=(\frac{7}{6},0,\frac{7}{6},\frac{113}{6})$.\\

    Stating this in terms of $\mathcal{B}$ and $\mathcal{L}$, we get
    $\bar{x}=(x_1,x_3,x_4,x_2)$, where $x_\mathcal{B}=(x_1,x_3,x_4)$ (since
    these $x_i$ were nonzero), and $x_\mathcal{L}=(x_2)$.\\

    Then, $\mathcal{B},\mathcal{L}$ are the following matrices:\\
    \begin{equation}
      \begin{split}
        \mathcal{B}&=
          \begin{bmatrix*}
            4 & -1 & 0 \\
            6 & 0 & 0 \\
            1 & 0 & 1 \\
          \end{bmatrix*} \\
        \mathcal{L}&=
          \begin{bmatrix*}
            -1 \\
            0 \\
            0 \\
          \end{bmatrix*}
      \end{split}
    \end{equation}

    Further, we know that $\pi^T=c^T_\mathcal{B}\mathcal{B}^{-1}$:\\
    \begin{equation}
      \begin{split}
        \pi^T&=
          \begin{bmatrix*}-3 & 0 & 0 \\\end{bmatrix*}
          \begin{bmatrix*}
            4 & -1 & 0 \\
            6 & 0 & 0 \\
            1 & 0 & 1 \\
          \end{bmatrix*} \\
        &=
          \begin{bmatrix*}
            15 \\
            0 \\
            3 \\
          \end{bmatrix*}
      \end{split}
    \end{equation}

    So the corresponding $\pi$s are $(\pi_1,\pi_2,\pi_3)=(15,0,3)$.\\
  % #2
  \item
    \begin{enumerate}
      \item
        % 2.a
        \begin{enumerate}
          \item
            The optimal solution is $(x_1,x_2)=(1.25,1.5)$.\\
            (see ``problem2a.m'').\\
          \item
            The dual is
            \begin{equation}
              \begin{split}
                \text{min}\ &5\pi_1+3\pi_2+24\pi_3+9\pi_4\\
                \text{s.t.}\ &\pi_1+6\pi_2+5\pi_3\geq1\\
                &\pi_1-3\pi_2+6\pi_4\geq 2\\
                &\pi_1,\pi_2\pi_3\pi_4\geq0\\
              \end{split}
            \end{equation}

            And the complementary slackness conditions are
            \begin{equation}
              \begin{split}
                x_1=0&\ \text{or}\ \pi_1+6\pi_2+5\pi_3=1\\
                x_2=0&\ \text{or}\ \pi_1-3\pi_2+6\pi_4=2\\
                \pi_1=0&\ \text{or}\ x_1+x_2=5\\
                \pi_2=0&\ \text{or}\ 6x_1-3x_2=3\\
                \pi_3=0&\ \text{or}\ 5x_1=24\\
                \pi_4=0&\ \text{or}\ 6x_2=9\\
              \end{split}
            \end{equation}
          \item
            Given $x^* = (1.25, 1.5)$, the complementary slackness conditions
            for the $\pi$s become:
            \begin{equation}
              \begin{split}
                \pi_1=0&\ \text{or}\ 2.75=5\\
                \pi_2=0&\ \text{or}\ 3=3\\
                \pi_3=0&\ \text{or}\ 6.25=24\\
                \pi_4=0&\ \text{or}\ 9=9\\
              \end{split}
            \end{equation}
            From this, we know that $\pi_1,\pi_3=0$. Now the complementary
            slackness conditions for the $x$s become:
            \begin{equation}
              \begin{split}
                x_1=0&\ \text{or}\ 6\pi_2=1\\
                x_2=0&\ \text{or}\ -3\pi_2+6\pi_4=2\\
              \end{split}
            \end{equation}
            Since $x_1, x_2\geq0$, $\pi_2,\pi_4$ must satisfy the equations on
            the right. Solving these gives
            $\pi^*=(0,\frac{1}{6},0,\frac{5}{12})$.\\
        \end{enumerate}
      \item
        % 2.b
        \begin{enumerate}
          \item
            The optimal solution is $(x_1,x_2,x_3)=(0,0,3)$\\
            (see the code in ``problem2b.m'').
          \item
            The dual is
            \begin{equation}
              \begin{split}
              \text{max}\ &-5\pi_1+6\pi_2\\
              \text{s.t.}\ &-\pi_1+\pi_2\leq 1\\
              &-2\pi_1\leq 0\\
              &2\pi_2\leq 1\\
              & \pi_1\geq 0, \pi_2\ \text{unconstrained}\\
              \end{split}
            \end{equation}

            And the complementary slackness conditions are
            \begin{equation}
              \begin{split}
                x_1=0&\ \text{or}\ \pi_1+\pi_2=1\\
                x_2=0&\ \text{or}\ 2\pi_1=0\\
                x_3=0&\ \text{or}\ 2\pi_2=6\\
                \pi_1=0&\ \text{or}\ x_1+2x_2=5\\
                \pi_2=0&\ \text{or}\ x_1+2x_3=6\\
              \end{split}
            \end{equation}
          \item
            Given $x^* = (0,0,3)$, the complementary slackness conditions
            for the $\pi$s become:
            \begin{equation}
              \begin{split}
                \pi_1=0&\ \text{or}\ 0=5\\
                \pi_2=0&\ \text{or}\ 6=6\\
              \end{split}
            \end{equation}
            From this, we know that $\pi_1=0$. Now the complementary slackness
            conditions for the $x$s become:
            \begin{equation}
              \begin{split}
                x_1=0&\ \text{or}\ \pi_2=1\\
                x_2=0&\ \text{or}\ 0=0\\
                x_3=0&\ \text{or}\ 2\pi_2=6\\
              \end{split}
            \end{equation}
            Since $x_3\ne0$, and the first and third equations cannot both be
            satisfied, we let the first equation go unsatisfied, and find
            $\pi^*=(0, 3)$.\\
        \end{enumerate}
      \item
        % 2.c
        \begin{enumerate}
          \item
            The optimal solution is $(x_1,x_2)=(1.1667,0)$.\\
            (see ``problem2c.m'').\\
          \item
            The dual is
            \begin{equation}
              \begin{split}
                \text{min}\ &-3\pi_1+7\pi_2+20\pi_3\\
                \text{s.t.}\ &-4\pi_1+6\pi_2+\pi_3\geq3\\
                &-5\pi_1-6\pi_2+8\pi_3\geq-5\\
                &\pi_1,\pi_3\geq0,\pi_2\ \text{unconstrained}\\
              \end{split}
            \end{equation}

            And the complementary slackness conditions are
            \begin{equation}
              \begin{split}
                x_1=0&\ \text{or}\ 4\pi_1+6\pi_2+\pi_3=3\\
                x_2=0&\ \text{or}\ 5\pi_1-6\pi_2+8\pi_3=-5\\
                \pi_1=0&\ \text{or}\ 4x_1+5x_2=3\\
                \pi_2=0&\ \text{or}\ 6x_1-6x_2=7\\
                \pi_3=0&\ \text{or}\ x_1+8x_2=20\\
              \end{split}
            \end{equation}
          \item
            Given $x^* = (\frac{7}{6},0)$, the complementary slackness conditions
            for the $\pi$s become:
            \begin{equation}
              \begin{split}
                \pi_1=0&\ \text{or}\ \frac{14}{3}=3\\
                \pi_2=0&\ \text{or}\ 7=7\\
                \pi_3=0&\ \text{or}\ \frac{7}{6}=20\\
              \end{split}
            \end{equation}
            From this, we know that $\pi_1,\pi_3=0$. Now the complementary slackness
            conditions for the $x$s become:
            \begin{equation}
              \begin{split}
                x_1=0&\ \text{or}\ 6\pi_2=3\\
                x_2=0&\ \text{or}\ -6\pi_2=-5\\
              \end{split}
            \end{equation}
            Since $x_2=0$, we can leave the second equation unsatisfied, and
            find $\pi^*=(0,\frac{1}{2},0)$.\\
        \end{enumerate}
      \item
        %2.d
        \begin{enumerate}
          \item
            The primal solution is infeasible, according to octave. (problem2d.m)
          \item
            The dual is
            \begin{equation}
              \begin{split}
                \text{min}\ &-8\pi_1+23\pi_2-\pi_3+4\pi_4\\
                \text{s.t.}\ &5\pi_1+4\pi_2-6\pi_3+\pi_4=3\\
                &3\pi_1+2\pi_2-7\pi_3=2\\
                &\pi_1+8\pi_2-3\pi_3\geq5\\
                &\pi_2,\pi_3,\pi_4\geq0,\pi_1\ \text{unconstrained}\\
              \end{split}
            \end{equation}

            And the complementary slackness conditions are
            \begin{equation}
              \begin{split}
                x_1=0&\ \text{or}\ 5\pi_1+4\pi_2+6\pi_3+\pi_4=3\\
                x_2=0&\ \text{or}\ 3\pi_1+2\pi_2+7\pi_3=2\\
                x_3=0&\ \text{or}\ \pi_1+8\pi_2+3\pi_3+\pi_5=5\\
                \pi_1=0&\ \text{or}\ 5x_1+3x_2+x_3=-8\\
                \pi_2=0&\ \text{or}\ 4x_1+2x_2+8x_3=23\\
                \pi_3=0&\ \text{or}\ 6x_1+7x_2+3x_3=1\\
                \pi_4=0&\ \text{or}\ x_1=4\\
                \pi_5=0&\ \text{or}\ x_3=0\\
              \end{split}
            \end{equation}
          \item
            The primal is infeasible, so skipping this step.\\
        \end{enumerate}
      \item
        % 2.e
        \begin{enumerate}
          \item
            Taking the third equation, we get:\\
            $x_1=1-x_2-x_3$

            Plugging $x_1$ into the inequalities gives us the following
            system of inequalities. Note that since the objective function
            does not depend on $x_i,i=1,2,3$, any feasible solution is optimal.

            \begin{equation}
              \begin{split}
                1+x_2-\frac{1}{2}x_3&\leq c\\
                3-x_2-2x_3&\leq c\\
              \end{split}
            \end{equation}

            Rearranging the first inequality gives $x_3\geq-2c+2+2x_2$, which
            we can substitute into the second inequality, yielding
            $x_2\geq\frac{3c-1}{5}$. Putting this back into the first
            inequality yields $x_3\geq\frac{9-7c}{5}$. Going back to the
            first equation, if we take $x_2=\frac{3c-1}{5},x_3=\frac{9-7c}{5}$,
            we find that $x_1=\frac{4c-3}{5}$.

            So, a feasible - and therefore optimal - solution is
            $(\frac{4c-3}{5},\frac{3c-1}{5},\frac{9-7c}{5})$\\
          \item
            The dual is
            \begin{equation}
              \begin{split}
                \text{max}\ &-c\pi_1-c\pi_2+\pi_3\\
                \text{s.t.}\ &-\pi_1-3\pi_2+\pi_3\leq0\\
                &-2\pi_1-2\pi_2+\pi_3\leq0\\
                &-\frac{1}{2}\pi_1-\pi_2+\pi_3\leq0\\
                &\pi_1,\pi_2\geq0,\pi_3\ \text{unconstrained}\\
              \end{split}
            \end{equation}

            And the complementary slackness conditions are
            \begin{equation}
              \begin{split}
                x_1=0&\ \text{or}\ \pi_1+3\pi_2+\pi_3=0\\
                x_2=0&\ \text{or}\ 2\pi_1+2\pi_2+\pi_3=0\\
                x_3=0&\ \text{or}\ \frac{1}{2}\pi_1+\pi_2+\pi_3=0\\
                \pi_1=0&\ \text{or}\ x_1+3x_2+\frac{1}{2}x_3=c\\
                \pi_2=0&\ \text{or}\ 3x_1+2x_2+x_3=c\\
                \pi_3=0&\ \text{or}\ x_1+x_2+x_3=1\\
              \end{split}
            \end{equation}
          \item
            Given that we took the equality case for each inequality to find
            $x^*$, none of the $\pi_i$ need to be zero. However, there are no
            constraints preventing the $\pi_i$ from being zero. In fact,
            $\pi^*=(0,0,0)$ solves the complementary slackness conditions for
            the $\pi$s.
        \end{enumerate}
    \end{enumerate}
  % #3
  \item
    The dual is
    \begin{equation}
      \begin{split}
        \text{min}\ &60\pi_1+\pi_2+\pi_3+\pi_4+\pi_5+\pi_6\\
        \text{s.t.}\ &5\pi_1+\pi_2\geq30\\
        &10\pi_1+\pi_3\geq20\\
        &20\pi_1+\pi_4\geq100\\
        &30\pi_1+\pi_5\geq90\\
        &40\pi_1+\pi_6\geq160\\
        &\pi_1\pi_2\pi_3\pi_4\pi_5\pi_6\geq0\\
      \end{split}
    \end{equation}

    And the complementary slackness conditions are
    \begin{equation}
      \begin{split}
        x_1=0&\ \text{or}\ 5\pi_1+\pi_2=30\\
        x_2=0&\ \text{or}\ 10\pi_1+\pi_3=20\\
        x_3=0&\ \text{or}\ 20\pi_1+\pi_4=100\\
        x_4=0&\ \text{or}\ 30\pi_1+\pi_5=90\\
        x_5=0&\ \text{or}\ 40\pi_1+\pi_6=160\\
        \pi_1=0&\ \text{or}\ 5x_1+10x_2+20x_3+30x_4+40x_5=60\\
        \pi_2=0&\ \text{or}\ x_1=1\\
        \pi_3=0&\ \text{or}\ x_2=1\\
        \pi_4=0&\ \text{or}\ x_3=1\\
        \pi_5=0&\ \text{or}\ x_4=1\\
        \pi_6=0&\ \text{or}\ x_5=1\\
      \end{split}
    \end{equation}
  % #4
  \item
    Let $k$ be a parameter to the contractor problem, with the problem
    defined as follows:
    \begin{itemize}
      \item The $j\textsuperscript{th}$ district requires $r_j$ teams.
      \item If $i<k$, $c_i$ is a skilled contractor. Otherwise, $c_i$ is
        unskilled.
      \item Each district must have at least one skilled team.
      \item The $i\textsuperscript{th}$ contractor has $u_i$ teams.
      \item There are $m$ contractors and $n$ districts. $k\leq m$.
      \item Decision variables are $x_{ij}$, where $x_{ij}$ represents the
        number of teams sent from contractor $i$ to region $j$. Similarly,
        the costs are $c_{ij}$
    \end{itemize}
    The primal is therefore (in standard form):
    \begin{alignat*}{2}
      &\text{min}&\sum_{i=1}^{m}\sum_{j=1}^{n}c_{ij}x_{ij}&\\
      &\text{s.t.}&\sum_{i=1}^kx_{ij}\geq1;\quad&j=1,\ldots,n\\
      &&\sum_{i=1}^mx_{ij}\geq r_j;\quad&j=1,\ldots,n\\
      &&-\sum_{j=1}^nx_{ij}\geq-u_i;\quad&i=1,\ldots,m\\
    \end{alignat*}

    Since the primal has $m\times n$ decision variables and $2n+m$ constraints,
    the dual will have $2n+m$ decision variables and $m\times n$ constraints.
    Before stating the dual, I will first define the function $f$ to be:
    \begin{equation}
      f(x) =
        \begin{cases}
          1&\text{if}\ x=0\\
          0&\text{otherwise}\\
        \end{cases}
    \end{equation}
    Then, the dual can be written as:
    \begin{alignat*}{2}
      &\text{max}
        &\sum_{j=1}^n\pi_j+\sum_{j=1}^nr_j\pi_{n+j}+\sum_{i=1}^m-u_i\pi_{2n+i}\\
      &\text{s.t.}
        &f(i-j)\pi_j+\pi_{n+j}-f(i-j)\pi_{2n+i}\leq c_{ij};\quad
        &i=1,\ldots,k;j=1,\ldots,n\\
      &&\pi_{n+j}-f(i-j)\pi_{2n+i}\leq c_{ij};\quad
        &i=k+1,\ldots,m;j=1,\ldots,n\\
    \end{alignat*}

    Consequently, the complementary slackness conditions are:
    \begin{alignat*}{2}
      &\pi_j=0\ \text{or}\ 
        &\sum_{i=1}^kx_{ij}=1;\ &j=1,\ldots,n\\
      &\pi_{n+j}=0\ \text{or}\
        &\sum_{i=1}^mx_{ij}= r_j;\ &j=1,\ldots,n\\
      &\pi_{2n+i}=0\ \text{or}\
        &-\sum_{j=1}^nx_{ij}=-u_i;\quad&i=1,\ldots,m\\
      &x_{ij}=0\ \text{or}\
        &f(i-j)\pi_j+\pi_{n+j}-f(i-j)\pi_{2n+i}\leq c_{ij};\
        &i=1,\ldots,k;j=1,\ldots,n\\
      &x_{ij}=0\ \text{or}\
        &\pi_{n+j}-f(i-j)\pi_{2n+i}\leq c_{ij};\
        &i=k+1,\ldots,m;j=1,\ldots,n\\
    \end{alignat*}
  % #5
  \item
    \begin{enumerate}
      \item
        \begin{proof} Every basic feasible solution is integer.\\
          The matrix given by the problem is (after adding slack variables)\\
          \begin{align*}
            M &=
            \begin{bmatrix*}
              1 & 0 & 1 & 0 & 1 & 0 & 0 & 0 & 0 & 0\\
              0 & 0 & -1 & 1 & 0 & 0 & 0 & 0 & 0 & 0\\
              -1 & 1 & 0 & 0 & 0 & 0 & 0 & 0 & 0 & 0\\
              0 & 1 & 0 & 1 & 1 & 0 & 0 & 0 & 0 & 0\\
              1 & 0 & 0 & 0 & 0 & 1 & 0 & 0 & 0 & 0\\
              0 & 1 & 0 & 0 & 0 & 0 & 1 & 0 & 0 & 0\\
              0 & 0 & 1 & 0 & 0 & 0 & 0 & 1 & 0 & 0\\
              0 & 0 & 0 & 1 & 0 & 0 & 0 & 0 & 1 & 0\\
              0 & 0 & 0 & 0 & 1 & 0 & 0 & 0 & 0 & 1\\
            \end{bmatrix*}
          \end{align*}

          This is a $10\times9$ matrix, so there are 10 $9\times9$ submatrices
          that we need to test, each obtained by removing a column from the
          above matrix.\\
          Note that the top four rows are linearly dependent, by taking the sum
          of the first three rows and subtracting the fourth. Also note that
          this holds when any single column is removed. Therefore, all of these
          submatrices have a 0 determinant, so $M$ is unimodular, and so by
          Cramer`s rule, $Mx=b$, with $b$ integer, is satisfiable by an integer
          $x$. Since $b$ for this linear program is integer, any basic feasible
          solution is integer.\\
        \end{proof}
      \item
        See ``problem5b.m'' for the code. An optimal basic feasible solution
        is $(x_1,x_2,x_3,x_4,x_5)=(1,0,0,1,1)$.
      \item
        The dual is
        \begin{equation}
          \begin{split}
            \text{max}\ &2\pi_1+\pi_2-\pi_3+2\pi_4-\pi_5-2\pi_6-\pi_7-3\pi_8-2\pi_9\\
            \text{s.t.}\ &\pi_1-\pi_3-\pi_5\leq1\\
            &\pi_3+\pi_4-\pi_6\leq1\\
            &\pi_1-\pi_2-\pi_7\leq3\\
            &\pi_2+\pi_4-\pi_8\leq2\\
            &\pi_1+\pi_4-\pi_9\leq4\\
            &\pi_1,\pi_2\pi_3\pi_4\ \text{unconstrained},\pi_5,\pi_6\pi_7\pi_8,\pi_9\geq0\\
          \end{split}
        \end{equation}

        And the complementary slackness conditions are
        \begin{equation}
          \begin{split}
            x_1=0&\ \text{or}\ \pi_1-\pi_3+\pi_5=1\\
            x_2=0&\ \text{or}\ \pi_3+\pi_4+\pi_6=1\\
            x_3=0&\ \text{or}\ \pi_1-\pi_2+\pi_7=3\\
            x_4=0&\ \text{or}\ \pi_2+\pi_4+\pi_8=2\\
            x_5=0&\ \text{or}\ \pi_1+\pi_4+\pi_9=4\\
            \pi_1=0&\ \text{or}\ x_1+x_3+x_5=2\\
            \pi_2=0&\ \text{or}\ -x_3+x_4=1\\
            \pi_3=0&\ \text{or}\ -x_1+x_2=-1\\
            \pi_4=0&\ \text{or}\ x_2+x_4+x_5=2\\
            \pi_5=0&\ \text{or}\ -x_1=-1\\
            \pi_6=0&\ \text{or}\ -x_2=-2\\
            \pi_7=0&\ \text{or}\ -x_3=-1\\
            \pi_8=0&\ \text{or}\ -x_4=-3\\
            \pi_9=0&\ \text{or}\ -x_5=-2\\
          \end{split}
        \end{equation}
      \item
        Given $x^* = (1,0,0,1,1)$, the complementary slackness conditions
        for the $\pi$s become:
        \begin{equation}
          \begin{split}
            \pi_1=0&\ \text{or}\ 2=2\\
            \pi_2=0&\ \text{or}\ 1=1\\
            \pi_3=0&\ \text{or}\ -1=-1\\
            \pi_4=0&\ \text{or}\ 2=2\\
            \pi_5=0&\ \text{or}\ -1=-1\\
            \pi_6=0&\ \text{or}\ 0=-2\\
            \pi_7=0&\ \text{or}\ 0=-1\\
            \pi_8=0&\ \text{or}\ -1=-3\\
            \pi_9=0&\ \text{or}\ -1=-2\\
          \end{split}
        \end{equation}
        From this, we know that $\pi_6,\pi_7,\pi_8,\pi_9=0$. Now the
        complementary slackness conditions for the $x$s become:
        \begin{equation}
          \begin{split}
            x_1=0&\ \text{or}\ \pi_1-\pi_3+\pi_5=1\\
            x_2=0&\ \text{or}\ \pi_3+\pi_4=1\\
            x_3=0&\ \text{or}\ \pi_1-\pi_2=3\\
            x_4=0&\ \text{or}\ \pi_2+\pi_4=2\\
            x_5=0&\ \text{or}\ \pi_1+\pi_4=4\\
          \end{split}
        \end{equation}
        We can also remove the equations for which $x_2,x_3=0$, so:
        \begin{equation}
          \begin{split}
            x_1=0&\ \text{or}\ \pi_1-\pi_3+\pi_5=1\\
            x_4=0&\ \text{or}\ \pi_2+\pi_4=2\\
            x_5=0&\ \text{or}\ \pi_1+\pi_4=4\\
          \end{split}
        \end{equation}
        We can now pick the $\pi_i$s. If we let $\pi_1=4,\pi_2=2,\pi_3=3$ and
        $\pi_4,\pi_5=0$, the equations are satisfied.\\

        Thus, $\pi^*=(4,2,3,0,0,0,0,0,0)$ satisfies the complementary
        slackness conditions.\\
    \end{enumerate}
  % #6
  \item
    \begin{enumerate}
      \item
        Let the objective function of the linear program be $z(x)=0$. Then
        $A$ is defined by the coefficients of the $m$ linear inequalities, and
        $b_i$ is the right-hand-side of the $i$\textsuperscript{th} linear
        inequality.\\

        Define $LP$ to be the linear program which minimizes $z(x)$ subject to
        the constraints imposed by $A$ and $b$ (without loss of generality,
        define $A,b$ such that the original $m$ inequalities are all
        lower-bounds, by multiplying by $-1$ or adding slack variables where
        appropriate). That is, LP is
        \begin{equation}
          \begin{split}
            \text{min}\ &z(x)=0\\
            \text{min}\ &Ax\geq b\\
          \end{split}
        \end{equation}
        Note that because the objective function is 0 (any constant will do,
        actually), it does not depend on $x$, so any feasible solution is
        optimal. In this way, if we apply our linear programming algorithm to
        $LP$, and it finds \textit{any} solution, then the original system of
        inequalities can be satisfied. If the algorithm cannot find a solution,
        then the system of inequalities cannot be satisfied.
      \item
        Begin with the inputs to a linear program:
          \begin{itemize}
            \item $A$: the matrix of coefficients
            \item $b$: the vector of constraints
            \item $c$: the coefficients of the objective function
          \end{itemize}
        With the LP in standard form ($Ax=b, x\geq0$).\\

        Let L.I.F. be the algorithm which solves the linear inequality
        feasibility problem. That is L.I.F.($A, b$) returns a vector $x$
        where $Ax\geq b$ or nil if $Ax\geq b$ is unfeasible.\\

        If we compute L.I.F.($A, b$), this will give us \textit{a} feasible
        solution to the LP, but it is not necessarily optimal, since $c$ is not
        considered at all. However, if we take advantage of duality, we can
        construct a LIF problem which solves both the primal and the dual of
        the LP simultaneously, or not at all if there is no solution.\\

        By weak duality, we know that for any feasible $x,\pi$ ($x$ feasible
        in the primal, and $\pi$ in the dual), the following holds:
        \begin{equation}
          \pi^Tb\leq\pi^TAx\leq c^Tx
        \end{equation}

        Further, by strong duality, the inequalities become strict equality
        when $x, \pi$ are respectively optimal. That is,
        \begin{equation}
          \begin{split}
            \pi^Tb=\pi^TAx=c^Tx\\
            \pi^Tb=c^Tx\\
          \end{split}
        \end{equation}

        To express these strict equality constraints ($i=j$) in the LIF
        problem, we need two constraints, $i\geq j$ and $-i\geq j$. This
        enforces an equality constraint.\\

        So, if we define the following matrix $\bar{A}$ and vector $\bar{b}$:
        \begin{equation}
          \begin{split}
            \bar{A}&=
              \begin{bmatrix*}
                A & 0 \\
                -A & 0 \\
                I & 0 \\
                0 & A^T \\
                0 & -A^T \\
                0 & I \\
                -c^T & b^T \\
                c^T & -b^T
              \end{bmatrix*}\\
            \bar{b}&=
              \begin{bmatrix*}
                b \\ -b \\ \vec{0} \\ c \\ -c \\ \vec{0} \\ 0 \\ 0
              \end{bmatrix*}
          \end{split}
        \end{equation}

        We can now get the result of $\text{LIF}(\bar{A},\bar{b})$, $y$.\\
        $y$ satisfies $\bar{A}y\geq\bar{b}$, and further, $y=(x^*,\pi^*)$.\\

        Consider the first three rows of $\bar{A}y\geq\bar{b}$. This expresses
        a feasible solution to the LP: $Ax=b,x\geq0$.\\

        Now consider the next three rows of $\bar{A}y\geq\bar{b}$. This
        expresses a feasible solution to the LP: $A^T\pi=c,\pi\geq0$, which is
        also a feasible solution to the dual of the LP expressed by the first
        three rows.\\

        The last two rows express the strong duality constraint that, if
        $x,\pi$ optimal are respectively feasible to the primal and the dual,
        then $\pi^Tb=c^Tx$.\\

        All together, $\bar{A}y\geq\bar{b}$ yields $y=(x^*,\pi^*)$, where
        $x^*$ is feasible in the primal, $\pi^*$ is feasible in the dual, and
        $x^*,\pi^*$ satisfy strong duality.\\

        If LIF can find a $y$ which satisfies $\bar{A}y\geq\bar{b}$, then
        $y_i,i=1,2\ldots,q$ is an optimal solution to the primal, and
        $y_i,i=q+1,q+2\ldots,q+p$ is an optimal solution to the dual. If LIF
        cannot find such a $y$ then at least one of the primal or the dual is
        infeasible or unbounded.\\
    \end{enumerate}
\end{enumerate}
\end{document}
